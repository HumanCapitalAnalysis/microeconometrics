%-------------------------------------------------------------------------------
%-------------------------------------------------------------------------------
\begin{frame} I heavily draw on the material presented in:

\begin{itemize}
\item Whitney Newey, course materials for 14.385 Nonlinear Econometric Analysis, Fall 2007. MIT OpenCourseWare
(http://ocw.mit.edu), Massachusetts Institute of Technology.
\end{itemize}

\end{frame}
%-------------------------------------------------------------------------------
%-------------------------------------------------------------------------------
\begin{frame}\textbf{General idea}\vspace{0.3cm}
\begin{itemize}\setlength\itemsep{1em}
\item The generalized method of moments (GMM) is a general estimation principle, where the estimators are derived from so-called  moment conditions. It provides a unifying framework for the comparison of alternative estimators.
\end{itemize}

\end{frame}
%-------------------------------------------------------------------------------
%-------------------------------------------------------------------------------
\begin{frame}\textbf{Structure}\vspace{0.3cm}

\begin{itemize}\setlength\itemsep{1em}
\item Setup
\item Identification
\item Asymptotic distribution
\item Testing
\end{itemize}
\end{frame}
%-------------------------------------------------------------------------------
%-------------------------------------------------------------------------------
